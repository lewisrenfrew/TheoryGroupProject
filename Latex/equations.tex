% simple.tex - a simple article to ilustrate document structure 


\documentclass{article}
\usepackage{graphicx}


\begin{document}

\title{ Equations for Project work \LaTeX{} script }
\author{Olle Windeman\\
  School of physics,\\
  university of Glasgow.\\
  Glasgow,\\
  scotland,\\
  \texttt{olle@Windeman.se}}
\date{\today}
\maketitle



\section{Equations}
gauss' law
\begin{equation}
\vec{\nabla} \cdot \vec{E} = \frac{\rho}{\varepsilon_{0}}
\label{1}
\end{equation}

for no enclosed charge
\begin{equation}
\vec{\nabla} \cdot \vec{E} = 0
\label{2}
\end{equation}

and the electric field
\begin{equation}
\vec{E} = -\vec{\nabla}\phi
\label{3}
\end{equation}

using eq3 in eq2 gives
\begin{equation}
\vec{\nabla} \cdot (-\vec{\nabla}\phi) = 0
\label{4}
\end{equation}

which give the laplacian
\begin{equation}
\vec{\nabla}^2\phi = 0
\label{5}
\end{equation}

the laplacian in polar coordinates
\begin{equation}
\vec{\nabla}^2\phi = \frac{\partial^2\phi}{\partial r^2}+\frac{1}{r}\frac{\partial\phi}{\partial r}+\frac{1}{r^2} \frac{\partial^2\phi}{\partial\theta^2}=0
\label{6}
\end{equation}

the function is the same at all theta so last term disapears
\begin{equation}
\frac{1}{r^2} \frac{\partial^2\phi}{\partial\theta^2}=0
\label{7}
\end{equation}

this gives
\begin{equation}
\vec{\nabla}^2\phi = \frac{\partial^2\phi}{\partial r^2}+\frac{1}{r}\frac{\partial\phi}{\partial r}=0
\label{8}
\end{equation}

moving the r
\begin{equation}
\vec{\nabla}^2\phi = r\frac{\partial^2\phi}{\partial r^2}+\frac{\partial\phi}{\partial r}=0
\label{9}
\end{equation}

set variable lambda
\begin{equation}
\lambda(r) = \frac{\partial\phi}{\partial r}
\label{10}
\end{equation}

this gives a characteristic equation
\begin{equation}
r\lambda ' + \lambda = 0
\label{11}
\end{equation}

this gives
\begin{equation}
r\frac{\partial \lambda}{\partial r} = -\lambda
\label{12}
\end{equation}

moving the partials
\begin{equation}
\frac{\partial \lambda}{\lambda}=-\frac{\partial r}{r}
\label{13}
\end{equation}

integrating both sides
\begin{equation}
\int\frac{\partial \lambda}{\lambda}=-\int\frac{\partial r}{r}
\label{14}
\end{equation}

That gives, c1 is in the reals
\begin{equation}
ln|\lambda(r)|=-ln|r| + c_1
\label{15}
\end{equation}


take the exponential on both sides
\begin{equation}
\lambda(r)=\frac{1}{r} + c_2
\label{16}
\end{equation}

where c2 is
\begin{equation}
c_2=e^{c_1}
\label{17}
\end{equation}

going back from lambdas
\begin{equation}
\phi(r)=\int\lambda(r)dr
\label{18}
\end{equation}

this gives
\begin{equation}
\phi(r)=c_2 \int\frac{1}{r}dr
\label{19}
\end{equation}

integrating gives
\begin{equation}
\phi(r)=c_2ln|r| + c_3
\label{20}
\end{equation}

boundary condition 1 , Ri is r inner
\begin{equation}
\phi(r=R_i)=0
\label{21}
\end{equation}

boundary condition 2 , Ro is r outer
\begin{equation}
\phi(r=R_o)=V
\label{22}
\end{equation}

for boundary 1
\begin{equation}
\phi(R_i)=c_2ln|R_i| + c_3=0
\label{23}
\end{equation}

this leads to
\begin{equation}
c_3=-c_2ln|R_1|
\label{24}
\end{equation}

putting \ref{24} into \ref{20}
\begin{equation}
\phi(r)=c_2ln|r| - c_2ln|R_1| 
\label{25}
\end{equation}

Then \ref{25} becomes
\begin{equation}
\phi(r)=c_2ln\left|\frac{r}{R_i}\right|
\label{26}
\end{equation}

using 2 bounuary condition on \ref{26} gives
\begin{equation}
\phi(R_o)=c_2ln\left|\frac{R_o}{R_i}\right|=V
\label{27}
\end{equation}

finding a value for c2
\begin{equation}
c_2=\frac{V}{ln\left|\frac{R_o}{R_i}\right|}
\label{28}
\end{equation}

so the final function can be found putting \ref{28} into \ref{26}
\begin{equation}
\phi(r)=\frac{V}{ln\left|\frac{R_o}{R_i}\right|}ln\left|\frac{r}{R_i}\right|
\label{29}
\end{equation}


\end{document}
